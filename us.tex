\documentclass[kpfonts]{patmorin}
\listfiles
\usepackage{pat}
\usepackage{paralist}
\usepackage{dsfont}  % for \mathds{A}
\usepackage[utf8x]{inputenc}
\usepackage{skull}

\usepackage{graphicx}
\usepackage[noend]{algorithmic}

\usepackage{xcolor}
\definecolor{light-gray}{gray}{0.95}

\usepackage[normalem]{ulem}
\usepackage{cancel}
\usepackage{enumitem}

\usepackage[longnamesfirst,numbers,sort&compress]{natbib}

\usepackage{cleveref}

\newcommand{\snote}[1]{\fcolorbox{red}{yellow}{#1}}
\newcommand{\pnote}[1]{\ \newline\noindent\fcolorbox{red}{yellow}{\begin{minipage}{\textwidth}#1\end{minipage}}}
\setlength{\parskip}{1ex}

\title{\MakeUppercase{(Layered) Partitions versus Decompositions}}
\author{}

\DeclareMathOperator{\tw}{tw}
\DeclareMathOperator{\ltw}{ltw}
\DeclareMathOperator{\pw}{pw}
\DeclareMathOperator{\lpw}{lpw}
\DeclareMathOperator{\lhptw}{lhp-tw}
\DeclareMathOperator{\lhppw}{lhp-pw}


\title{\MakeUppercase{Optimal 2-Ranking of Planar Graphs}}
\author{Working Group}

\newcommand{\uqs}{\chi_2}
\newcommand{\hus}{\hat{\chi}_2}

\newtheorem{othertheorem}{Theorem}
\renewcommand*{\theothertheorem}{\Alph{othertheorem}}
\crefname{othertheorem}{Theorem}{Theorem}

\begin{document}
% \begin{titlepage}
\maketitle

\begin{abstract}
  A 2-ranking is a labelling of the vertices of a graph with integer colours so that the maximum label along any path of length at most 2 is unique.  We show that every $n$-vertex planar graph has a unique-superior colouring using $O(\log n/\log\log n)$ colours.
\end{abstract}
% \end{titlepage}

% \tableofcontents

% \newpage
% \pagenumbering{arabic}

\section{Introduction}

% A sequence $s_0,\ldots,s_p$ over some total order $(S,<)$ is \emph{ranked} if $\max\{s_0,\ldots,s_p\}$ occurs exactly once in $s_0,\ldots,s_p$.

A colouring of a graph $G$ is a function $\varphi:V(G)\to \N$.  A colouring $\varphi$ is an \emph{$\ell$-ranking} if, for every path $v_0,\ldots,v_r$ of length $r\le\ell$, $\max\{\varphi(v_0),\ldots,\varphi(v_r)\}$ appears exactly once in the sequence $\langle \varphi(v_0),\ldots,\varphi(v_r)\rangle$.\footnote{An alternative, but equivalent condition is that for every path $v_0,\ldots,v_r$ of length $r\le\ell$,
\begin{inparaenum}[(i)]
   \item $\varphi(v_0)\neq \varphi(v_r)$; or
   \item $\max\{\varphi(v_0),\ldots,\varphi(v_r)\} > \varphi(v_0)$.
\end{inparaenum}
}
The $\ell$-ranking number $\chi_\ell(G)$ is the minimum integer $k$ such that $G$ has a $\ell$-ranking $\varphi:V(G)\to \{1,\ldots,k\}$.

The case $\ell=2$ has received special attention \cite{almeter.demircan.ea:graph,karpas.neiman.ea:on}. A $2$-ranking is called a \emph{unique-superior colouring} (abbreviated \emph{us-colouring}) by \citet{karpas.neiman.ea:on} who prove the following result:

\setcounter{othertheorem}{19}
\begin{othertheorem}[\cite{karpas.neiman.ea:on}]\label{trees}
    For every $n$-vertex tree $T$, $\uqs(T)\in O(\log n/\log\log n)$ and this is asymptotically optimal: for every $n\ge 1$ there exists an $n$-vertex tree $T$ with $\uqs(T)\in\Omega(\log n/\log\log n)$.
\end{othertheorem}

The same authors prove the following result for planar graphs:

\setcounter{othertheorem}{15}
\begin{othertheorem}[\cite{karpas.neiman.ea:on}]\label{planar-graphs}
    For every $n$-vertex planar graph $G$, $\uqs(G)\in O(\log n)$.
\end{othertheorem}

Since every tree is a planar graph and no better lower bound is known for planar graphs, this leaves an obvious question:  Which is the correct bound for planar graphs $\log n$ or $\log n/\log\log n$?  As it turns out, neither is correct.  Let $\log x =\max\{2,\log x\}$, let $\log^{(0)}x:=x$ and, for any integer $i>0$, let $\log^{(i)}x:=\log(\log^{(i-1)} x)$. We prove:


\begin{thm}\label{planar}
    For every $n$-vertex planar graph $G$, $\uqs(G)\in O(\log n/\log^{(4)} n)$.  For every $n\ge 1$ there exists an $n$-vertex planar graph $G$ with $\uqs(G)\in O(\log n/\log^{(3)} n)$
\end{thm}

Our proof of \cref{planar} makes use of a recent \emph{product structure theorem} of \citet{dujmovic.joret.ea:planar} which states that every planar graph $G$ is a subgraph of $H\boxtimes K_3\boxtimes P$ where $H$ is a planar $3$-tree, $K_3$ is a 3-cycle, $P$ is a path, and $\boxtimes$ denotes the strong graph product.\footnote{A $k$-tree is an edge-maximal graph of treewidth $k$. Formal definitions of treewidth and strong graph product appear later.}  To apply this theorem, we generalize \cref{trees} as follows:

\begin{thm}\label{t-trees}
    For every fixed $t$ and every $n$-vertex $t$-tree, $\uqs(G) \in O(\log n/\log^{(t+1)}(n)$ and this is asymptotically optimal: for every $n\ge 1$ there exists an $n$-vertex $t$-tree $T$ with $\uqs(T)\in\Omega(\log n/\log^{(t+1)} n)$.
\end{thm}

\cref{t-trees} is then applied to the graph $H$ along with a simple product colouring which shows that $\uqs(G_1\boxtimes G_2)\le \uqs(G_1)\cdot\uqs(G_2)$. It is easy to see that $\uqs(K_3)=3$ and $\uqs(P)\le 3$, so this implies that $\uqs(H\boxtimes P)\le 9\uqs(H)$.

% The remainder of this paper is organized as follows: \Cref{sec:trees} reviews the proof of \cref{trees}. \Cref{sec:t-trees} proves \Cref{t-trees}.  \Cref{sec:products} proves \Cref{planar} as well its generalization to several other graph classes, including graphs of bounded genus, apex-minor-free graphs, bounded-degree graphs from minor-closed families, and $k$-planar graphs.


% \section{A Restatement of the Proof for Trees}
% \seclabel{trees}
%
% In this section we review a slight variant of the proof of \cref{trees} by \citet{karpas.neiman.ea:on}. This presentation modifies the proof slightly so that it separates the treatment of the two different types length-$2$ paths in trees. This will simplify the arguments in \cref{sec:t-trees}.
%
% Let $H$ be a rooted tree.  The \emph{depth} $d_H(V)$ of a node $v\in H$ is the length of the path $P_T(v)$ from the root $r$ of $H$ to $v$.  A node $a\in V(T)$ is a $T$-ancestor of $w\in V(T)$ if $a\in V(P_T(w))$.  A $T$-ancestor $a$ of $w$ is \emph{strict} if $a\neq w$.  If $a$ is a (strict) $T$-ancestor of $w$ then $w$ is a (strict) $T$-descendant of $a$.  A path $P$ in $H$ is \emph{vertical} if no two nodes in $P$ have the same depth, otherwise the path is \emph{humped}.  A \emph{humped} us-colouring of $H$ is a proper colouring of $H$ in which, for every humped length-$2$ path $uvw$ in $H$, at least one of $\varphi(u)$ or $\varphi(w)$ is not equal to $\max\{\varphi(u),\varphi(v),\varphi(w)\}$.  For a rooted tree $H$, we let $\hus(H)$ denote the \emph{humped us-chromatic number} of $H$; the smallest integer $k$ such that $H$ has a humped us-colouring $\varphi:V(H)\to\{1,\ldots,k\}$.
%
% \begin{lem}\label{product}
%     For any rooted tree $H$, $\uqs(H)\le 3\hus(H)$.
% \end{lem}
%
% \begin{proof}
%     It is slightly more convenient to show the equivalent statement: If has a humped unique-superior colouring $\varphi:V(H)\to S$, then $H$ has a unique-superior colouring $\varphi':V(H)\to S\times\{0,1,2\}$, where the elements of $S\times\{0,1,2\}$ are ordered lexicographically.
%
%     For each $v\in V(H)$, let $\varphi'(v):=(\varphi(v), d_H(v)\bmod 3)$. Since $\varphi$ is a humped us-colouring of $H$ we need only consider vertical paths of $H$.  But this is trivial since for every vertical path $uvw$ in $H$, $\{\varphi'(u),\varphi'(v),\varphi'(w)\}=\{0,1,2\}$, so $\varphi'(u)$, $\varphi'(v)$, and $\varphi'(w)$ are distinct.
% \end{proof}
%
% For integers $0\le i\le k$, let $f_k(i)$ be the smallest integer such that there exists an $f_k(i)$-node tree $H$ with root $r$ and some value $\ell \in\{i,\ldots,k\}$ such that $H$ does not have a humped unique-superior colouring $c:V(G)\to \{1,\ldots,k\}$ (in which the root of $H$ happens to be coloured $k$).
%
% \begin{lem}\label{recursion}
%     For any integers $1\le  i\le k$, $f_k(i) \ge (k-i+1)\cdot f_k(i-1)$
%     \[
%         f_k(i) \le \begin{cases}
%                     k & \text{if $i=1$} \\
%                     (k-i+1)\cdot f(i-1) & \text{if $i>1$}
%                  \end{cases}
%     \]
% \end{lem}
%
% \begin{proof}
%     The proof is by induction on $i$. First consider the base case $i=1$ and let $H$ be a $f(1)$-node tree $H$ with root $r$ such that, for some $\ell\in\{1,\ldots,k\}$, $H$ does not have a a humped unique superior colouring $\varphi:V(G)\to \{1,\ldots,k\}$.  Let $r_1,\ldots,r_d$ be the children of $r$.
%     Then, for each $j\in\{1,\ldots,d\}$, and each $\ell'\in \{1,\ldots,k\}$, the subtree $H_j$ of $H$ rooted at $r_i$ has fewer than $f(1)$ nodes, so $H_j$ has a humped unique-superior colouring $\varphi:V(H_j)\to \{1,\ldots,k\}$ with $\varphi(r_j)=\ell'$.  Therefore, if $d \le k-1$, we can obtain a humped unique-superior colouring of $H$ as follows:
%     \begin{enumerate}
%         \item Set $\varphi(r):=\ell$.
%         \item For each $j\in\{1,\ldots,d\}$, assign each $r_j$ a unique colour  $\varphi(r_j)\in\{1,\ldots,k\}\setminus\{\ell\}$ and extend $\varphi$ to a humped unique-superior colouring $\varphi:V(H_j)\to\{1,\ldots,k\}$ of $H_j$.
%     \end{enumerate}
%     If $d\le k-1$, then $|\{1,\ldots,k\}-\{\ell\}|=k-1\ge d$, so Step~2 will succeed.
%     It is easy to check that $\varphi:V(H)\to\{1,\ldots,k\}$ is a humped unique-superior colouring of $H$ with $\varphi(r)=\ell$.  By definition, no such colouring exists, therefore it must be the case that $d> k-1$, so $H$ has at least $k+1$ vertices.
%
%     Next consider some $i\in\{2,\ldots,k\}$.  Let $r_1,\ldots,r_d$ be the children of $r$ ordered so that the subtrees $H_1,\ldots,H_{d'}$ each have size at least $f(i-1)$ and the subtrees $H_{d'+1},\ldots,H_d$ each have size less than $f(i-1)$.  If $d'\le k-i$, then we obtain a humped unique-superior colouring of $H$ as follows:
%     \begin{enumerate}
%         \item Set $\varphi(r):=\ell$.
%         \item For each $j\in\{d'+1,\ldots,d\}$ set $\varphi(r_j)=i-1$ and extend this to a humped unique-superior colouring of $H_j$.
%         \item For each $j\in\{1,\ldots,d'\}$, assign each $r_j$ a unique colour $\varphi(r_j)\in\{i,\ldots,k\}\setminus\{\ell\}$ and extend this to a humped unique-superior colouring of $H_j$.
%     \end{enumerate}
%     If $d'\le k-i$ then Step~3 is always possible since $|\{i,\ldots,k\}\setminus\{\ell\}|=k-i$. Again, it is easy to check that $\varphi:V(H)\to[k]$ is a humped unique-superior colouring of $H$ with $\varphi(r)=\ell$.  By definition, no such colouring exists, so it must be the case that $d'>k-i$. Therefore $H$ has at least $d'\cdot f_k(i-1)\ge (k-i+1)\cdot f_k(i-1)$ vertices.
% \end{proof}
%
% Unravelling the recurrence in \cref{recursion} shows that $f_k(k)\ge 1\cdot2\cdot\cdots k=k!$.  In particular, for any $n < f_k(k)=k!$, every $n$ vertex rooted tree $H$ has a humped unique-superior colouring $\varphi(H)\to\{1,\ldots,k\}$. Combining this with \cref{product} yields:
%
% \begin{thm}[\cite{karpas.neiman.ea:on}]\label{trees-precise}
%     For every tree $H$ with $n<k!$ vertices $\uqs(H)\le 3k$.
% \end{thm}
%
% Using Stirling's Approximation of $k!$ it easy to see that $n < k!$ for some $k\in\Theta(\log n/\log\log n)$, so \cref{trees} is a corollary of \cref{trees-precise}.
%
% \section{A Less Obvious Restatement of the Proof for Trees}
% \seclabel{trees}
%
% \begin{lem}\label{tree-separator}
%     For every integer $c\ge 1$ and for every $n$-vertex rooted tree $H$, there exists $S\subseteq V(H)$ such that
%     \begin{inparaenum}[(i)]
%         \item $H[S]$ is a tree that contains the root of $H$; \label{has-root}
%         \item $H[S]$ has at most $c$ leaves; and \label{c-1-leaves}
%         \item $H-S$ has no component larger than $n/(c+1)$. \label{max-component}
%     \end{inparaenum}
% \end{lem}
%
% \begin{proof}
%     Let $r$ be the root of $H$ and let $v_1,\ldots,v_n$ be the vertices of $H$ in the order they are encountered in a preorder traveral.  Let $L=\{v_{\lceil in/c\rceil}: i\in\{1,\ldots,c\}\}$.  Let $S:=\bigcup_{\ell\in L} V(P_H(\ell))$ so that $H[S]$ is the minimum spanning tree of $L\cup\{r\}$.  Clearly $S$ satisfies (\ref{has-root}).  Every leaf of $H[S]$ is contained in $L$, so $S$ satisifies (\ref{c-1-leaves}).  That $S$ satisfies (\ref{max-component}) follows from the fact that, for any $i\in\{1,\ldots,n\}$, $P_H(v_i)$ separates $v_1,\ldots,v_{i-1}$ from $v_{i+1},\ldots,v_n$ in the sense that, for every $1\le a\le b\le n$ the path from $a$ to $b$ in $H$ contains at least one vertex of $P_T(v_i)$.
% \end{proof}
%
% \begin{lem}\label{skinny-tree-colour}
%     Let $1\le \ell \le n$ be integers, let $C$ be an $(\ell+1)$-element set of integers, let $\alpha\in C$, and let $H$ be an $n$-node tree rooted at $r\in v(H)$ having at most $\ell$ leaves.  Then $H$ has a humped us-colouring $\varphi:V(H)\to C$ in which $\varphi(r)=\alpha$.
% \end{lem}
%
% \begin{proof}
%     The proof is by induction on $|V(H)|$. If $|V(H)|=1$ then $H$ has one node $r$ and the result is trivial.  Otherwise, let $r_1,\ldots,r_m$ be the children of $r$ and observe that $1\le m\le \ell$.  Set $\varphi(r):=\alpha$, as required and, for each $i\in\{1,\ldots,m\}$, set $\varphi(r_i)$ to the $i$th largest value in $C\setminus\{\alpha\}$, so that each child of $r$ is assigned a unique colour distinct from $r$. It is easy to check that this is a humped us-colouring of the star $H[\{r,r_1,\ldots,r_m\}]$ rooted at $r$ so all that remains is to colour the subtree $H_{r_i}$ rooted at $r_i$ for each $i\in\{1,\ldots,m\}$.  Applying induction to $H_{r_i}$ with $\alpha=\varphi(r_i)$ for each $i\in\{1,\ldots,m\}$ completes the proof.
% \end{proof}
%
%
% \begin{lem}\label{tree-algorithm}
%     Let $1\le c\le k$ be integers and let $H$ be an $n$-vertex tree with $n\le k!/c!$, rooted at $r\in V(H)$. Then $H$ has a humped us-colouring $\varphi:V(H)\to\{0,\ldots,k\}$ with $\varphi(r)=k-c$.
% \end{lem}
%
% \begin{proof}
%     The proof is by induction on $k-c$.  The base case, when $c=k$, occurs when $n\le k!/c!=1$ simply states that setting $\varphi(r)=0$ is gives a us-colouring of the 1-node tree whose only node is $r$.
%
%
%     % The base case, when $c=k$ occurs when $n= k!/(k-1)!=k$.  In this case, $H$ has at most $k$ nodes and therefore, at most $k-1$ leaves.  The result then follows by applying \cref{skinny-tree-colour} with $C=\{1,\ldots,k\}$ and $\alpha=1$.
%
%     For $1\le c<k$, let $S$ be the set described in \cref{tree-separator}.  Then $H[S]$ has at most $c$ leaves so, by \cref{skinny-tree-colour} with $\alpha=k-c$, $\varphi$ can be extended into a us-colouring $\varphi:S\to C$ of $H[S]$ using only the $c+1$ colours in $C:=\{k-c,\ldots,k\}$.  Now $H-S$ is a forest consisting of trees $H_{r_1},\ldots,H_{r_m}$ each having size at most $n/(c+1) \le k!/(c+1)!$ with roots $r_1,\ldots,r_m$, respectively.  For each $i\in\{1,\ldots,m\}$, apply induction on $H_{r_i}$ to extend $\varphi$ into a humped us-colouring $\varphi:V(H_{r_i})\to\{1,\ldots,k\}$ of $H_{r_i}$ in which $\varphi(r_i)=k-c$.  It is now straightforward to check that (because $\alpha(r_i)=k-c < k-c+1$ for each $i\in\{1,\ldots,m\}$) $\varphi$ is a humped us-colouring of $H$.
% \end{proof}
%
% Applying \cref{tree-algorithm} with $c=1$ and shifting $0,\ldots,k$ onto $1,\ldots,k+1$ gives:
%
% \begin{cor}\label{tree-algorithm-ii}
%     For every rooted tree $H$ with $n\le k!$ nodes, $\hus(H)\le k+1$.
% \end{cor}
%
% Combining \cref{tree-algorithm-ii} with \cref{product} gives the following
% \begin{thm}[\cite{karpas.neiman.ea:on}]
%     For every tree $H$ with $n\le k!$ vertices, $\uqs(H)\le 3(k+1)$.
% \end{thm}

\section{Preliminaries}

\subsection{Basics}

Graphs, $N_G(v)$, $G[L]$.
Rooted trees, $d_T(v)$, $T$-ancestors and $T$-descendants.

\subsection{Treewidth}

A graph $H$ is a \emph{$t$-tree} if $H$ is a clique of size at most $t$ or if it contains a vertex $v$ such that $H[N_H(v)]$ is a $t$-clique and $H-\{t\}$ is a $t$-tree.  This recursive definition of $t$-trees implies that there is a permutation $\pi_1,\ldots,\pi_n$ of $V(G)$ such that $\pi_1,\ldots,\pi_t$ form a clique and, for each $i\in\{1,\ldots,n\}$, $\pi_i$ is adjacent to exactly $\min\{i-1,t\}$ vertices among $\pi_1,\ldots,\pi_{i-1}$ which themselves form a clique, which we call the \emph{parent clique} of $\pi_i$.  We call $\pi_1,\ldots,\pi_n$ a \emph{construction order} for $H$.  (Note that the parent clique of any vertex $v_i$, $i\ge t+1$ is determined by the choice of the \emph{root clique} $v_1,\ldots,v_t$, in the same way that the parent of a vertex in a tree is determined by the choice of the root.)

A \emph{tree decomposition} of a graph $H$ is a sequence $\mathcal{T}:=(B_x:x\in V(T))$ of subsets of $V(H)$ called \emph{bags} indexed by the nodes of a tree $T$ and such that
 \begin{inparaenum}[(i)]
     \item for each $v\in V(H)$, $T[\{x\in V(T):v\in B_x\}]$ is connected; and
     \item for each $vw\in E(H)$, there exists some $x\in v(T)$ such that $\{v,w\}\subseteq B_x$.
\end{inparaenum}
The \emph{width} of $\mathcal{T}$ is $\max\{|B_x|:x\in V(T)\}-1$. The \emph{treewidth} of $H$ is the minimum width of any tree decomposition of $H$.

A \emph{tree decomposition }$\mathcal{P}:=(B_x:x\in V(P))$ us called a \emph{path decomposition} if $P$ is a path. The \emph{pathwidth} $\pw(H)$ of a graph $H$ is the minimum width of any path decomposition of $H$.

It is not difficult to see that every $t$-tree $H$ has treewidth $t$ and a tree-decomposition of $H$ can be constructed incrementally from a construction order $\pi_1,\ldots,\pi_n$ of $H$: Start with a tree $T$ that consists of a single node $r$ with bag $B_r:=\{\pi_1,\ldots,\pi_{t+1}\}$ and, for $i\gets t+2,\ldots,n$, let $x$ be the minimum-depth node of $T$ such that $\pi_i\in B_x$, and add a new leaf $\pi_i$ to $T$ adjacent to $x$ whose bag $B_{\pi_i}$ contains $\pi_i$ and its parent clique.  We call a tree-decomposition constructed this way a \emph{canonical} tree decomposition of $H$.

\subsection{The Strong Graph Product}

Definition of $\boxtimes$.  Statement of the product structure theorem.


\subsection{Inequalities for Iterated Logarithms}

For $x\ge 1$, $a\ge 0$, we have the ineqality,
\begin{equation}
    \log (x+a) = \log (x(1+a/x)) = \log x + \log(1+a/x) \le \log x + \log e^{a/x} = \log x + \frac{a\log e}{x} \enspace , \label{log-x-plus-a}
\end{equation}
where the inequality follows from the inequality $e^z \le 1+z$, valid for all $z\in\R$.

Let $\tau(0):=1$ and, for integers $i>0$, $\tau(i):=2^{\tau(i-1)}$. For any integer $i>0$ and any $x\ge \tau(i)$, \cref{log-x-plus-a} generalizes as follows (by induction on $i$):
\begin{equation}
    \log^{(i)}(x+a) \le \log^{(i)} x + \frac{a\log^i e}{\prod_{j=0}^{i-1}\log^{(j)} x} \label{logi-x-plus-a}
\end{equation}

In several places we have ratios involving iterated logarithms, in which case we make use of the following consequence of \cref{logi-x-plus-a}
\begin{equation}
    \frac{\log^{(i)} x+a}{\log^{(i)} x} \le 1 + \frac{a\log^i e}{\prod_{j=0}^{i}\log^{(j)} x} \enspace, \label{logi-ratio}
\end{equation}
which is again valid for all $x\ge \tau(i)$.

\section{Colouring $t$-Trees}

In this section we prove asymptotically tight bounds for the worst-case number of colours needed to us-colour $n$-vertex $t$-trees.

\subsection{The Upper Bound for $t$-Trees}

For the remainder of this section, $H$ is a $t$-tree on $n\ge t+1$ vertices, $\pi_1,\ldots,\pi_n$ is a construction order for $H$, and $\mathcal{T}:=(B_x:x\in V(T))$ is a canonical tree decomposition of $H$.  It is easy to see that every bag of $\mathcal{T}$ has size exactly $t+1$, $T$ is rooted at $r$, and
\begin{inparaenum}[(i)]\setcounter{enumi}{2}
  \item for each node $y$ of $T$ with parent $x$, $|B_y\setminus B_x|=1$.\label{three}
\end{inparaenum}
Note that this implies that $|V(T)|=n-t$, since $B_r$ contains $t+1$ vertices of $H$ and, by (\ref{three}), each $x\in V(T)\setminus\{r\}$ includes exactly one vertex of $H$ that is not in its parent.

For each $v\in V(T)$ let $x_T(v)$ be the minimum-depth node $x\in V(T)$ such that $v\in B_x$.  A path $v_i,\ldots,v_r$ in $H$ is \emph{vertical} if $x_T(v_{i-1})$ is a strict $T$-ancestor of $x_T(v_i)$, for each $i\in\{1,\ldots,r\}$. (Equivalently, $v_0,\ldots,v_r$ is a subsequence of $\pi_1,\ldots,\pi_n$.) Any path in $H$ that is not vertical is \emph{humped}.  This allows us to define the humped us-chromatic number $\hus(H)=\hus(H,\mathcal{T})$ as the minimum number of colours needed in a humped us-colouring of $H$.

\begin{lem}\label{humped}
    For any $t$-tree $H$, $\uqs(H)\le \binom{t+2}{t}\hus(H)$.
\end{lem}

\begin{proof}
    A lemma of \citet{pilipczuk.siebertz:polynomial} (see \cite[Lemma~13]{pilipczuk.siebertz:polynomial-arxiv}) shows that from any vertex $v\in V(H)$, the number of vertical paths of length at most $p$ that end at $v$ is at most $\binom{t+p}{t}$.  Applying this with $p=2$ implies that we can greedily colour $v_1,\ldots,v_n$ to obtain a colouring $\psi:V(H)\to \{1,\ldots,\binom{t+2}{t}\}$ so that the endpoints of any non-trivial vertical path\footnote{A path is non-trivial if its length is at least 1.} of length at most $2$ are assigned different colours.  Let $\varphi:V(H)\to\{1,\ldots,k\}$ be a humped us-colouring of $H$.  As in \cref{product}, the colouring $\varphi':V(H)\to\{1,\ldots,k\}\times \{1,2,3\}$ defined by $\varphi'(v)=(\varphi(v),\varphi(v))$ proves the result.
\end{proof}


\begin{lem}\label{pathwidth}
    For any graph $H$, $\uqs(H)\le 3\pw(H) + 1$
\end{lem}

\begin{proof}
    The proof is by induction on $\pw(H)$.  The base case $\pw(H)=0$ is trivial: In this case, $H$ contains no edges and can be coloured with $1 = 3\pw(H)+1$ colour.

    For $\pw(H)>1$, it is well known that $H$ contains a sequence of vertices $v_1,\ldots,v_m$  such that
    \begin{inparaenum}
        \item $H$ contains no edge $v_iv_j$ with $|i-j|>1$;
        \item $H$ contains no path $v_iw v_j$ with $|i-j|>1$; and
        \item $\pw(H-\{v_1,\ldots,v_m\})\le \pw(H)-1$.
    \end{inparaenum}
    We can therefore inductively colour $H-\{v_1,\ldots,v_m\}$ using colours $\{1,\ldots,3\pw(H)-2\}$ and then colour each $v_i$ with colour $3\pw(H)-1+i\bmod 3$.
\end{proof}

A node $x$ in a rooted tree $T$ is a \emph{branching node} if $x$ has at least two children.  Let $\Lambda(T)$ denote the set of branching nodes in a tree $T$. For a graph $H$ with a rooted tree decomposition $\mathcal{T}:=(B_x:x\in V(T))$, we define the $\mathcal{T}$-skeleton of $H$ as $H[\bigcup_{x\in \Lambda(T)} B_x]$.

\begin{lem}\label{skeleton-colour}
    Let $H$ be a graph having a width-$t$ tree decomposition $\mathcal{T}$, and let $\hat{H}$ be the $\mathcal{T}$-skeleton of $H$.  Then $\uqs(H)\le \uqs(\hat{H}) + 3t+1$.
\end{lem}

\begin{proof}
    Colour the vertices of $\hat{H}$ using the colours $3t+2,\ldots,\uqs(\hat{H})+3t+1$.  Next, observe that the graph $H-\hat{H}$ has pathwidth at most $t$ so, by \cref{pathwidth}, it can be us-coloured using colours $1,\ldots,3t+1$.  This gives the desired colouring of $H$ using colours $\{1,\ldots,\uqs(\hat{H})+3t+1\}$.
\end{proof}


\subsection{2-Trees}

As a warm-up, let's start with $2$-trees.  For now, we ignore floors and ceilings.

\begin{lem}\label{tree-slack}
Let $c,k$ be integers $2\le c< k$, let $T$ be a non-empty tree with root $r$,  let $\gamma:V(T)\to\N$ be an assignment of positive weights to the vertices of $T$ and define
\[
    \Gamma(r):=\sum_{v\in V(T)} \frac{(\log k)^k}{(\log \gamma(v))^{\gamma(v)}} \enspace .
\]
There exists a constant $a>0$ such that,
if
\begin{equation}
     \Gamma(r) \le \frac{(\log k)^k}{\left(\log \left(c+\tfrac{\log c}{\log\log c}\right)\right)^{c+\tfrac{\log c}{\log\log c}}} \enspace , \label{total-weight-i}
\end{equation}
and
\begin{equation}
    \phi\in\left\{a\left(k-c-\tfrac{\log c}{\log\log c}\right)+1,\ldots,a(k-c-1)\right\} \label{phi-bound}
\end{equation}
then $T$ has a humped us-colouring $\varphi:V(T)\to\{1,\ldots,a(k-c-1)\}$ such that
\begin{compactenum}[(P1)]
    \item $\varphi(r)=\phi$;
    \item for each $v\in V(T)$, $\varphi(v)\ge a\cdot(k-\gamma(v)-1)+1$.
\end{compactenum}
\end{lem}

\begin{proof}
    The proof is by induction on $|V(T)|$.  In the base case, $|V(T)|=1$, so we set $\varphi(r):=\phi$, as required by (P1). The only thing to check is that this satisfies (P2). To check this, observe that
    \[ \frac{(\log k)^k}{(\log\gamma(r))^{\gamma(r)}}\le \Gamma(r) \le \frac{(\log k)^k}{\left(\log \left(c+\tfrac{\log c}{\log\log c}\right)\right)^{c+\tfrac{\log c}{\log\log c}}}
    \]
    so $\gamma(r) \ge c+\tfrac{\log c}{\log\log c}$. Therefore, \cref{phi-bound} ensures that (P2) is satisfied.

    Now assume $|V(T)|\ge 2$.  For each node $v\in V(T)$, let $\Gamma(v):=\sum_{w\in D_T(v)} (\log k)^k/(\log\gamma(v))^\gamma(v)$.  We say that node $v$ of $T$ is \emph{heavy} if
    \[
        \Gamma(v)> \frac{(\log k)^k}{\left(\log\left(c+1+\tfrac{\log (c+1)}{\log\log (c+1)}\right)\right)^{c+1+\tfrac{\log (c+1)}{\log\log(c+1)}}} \enspace .
    \]
    With a bit of work, we can show that this implies that
    \[
        W(v) > \frac{2}{3c(c-1)\log^2 e}\times \frac{(\log k)^k}{(\log c)^c}
    \]

    Let $T'$ be the subtree of $T$ induced by the set of heavy nodes.  Since no leaf of $T'$ is a $T$-descendant of any other leaf, it follows from \cref{total-weight-i} that the number of leaves of $T'$ is at most $(3c(c-1)\log^2 e)/2= O(c^2)$.

    Therefore the skeleton $\hat{T}'$ of $T'$ has $O(c^2)$ nodes so, by \cref{trees}, $\uqs(\hat{T}')\in O(\log c/\log\log c)$. Therefore, by \cref{skeleton-colour}, $T'$ has a us-colouring $\varphi':V(T')\to \{a(k-c-1)-r,\ldots,a(k-c-1)\}\setminus\{\phi\}$ for some $r\in O(\log c/\log\log c)$.  We use $\varphi'$ to obtain a colouring $\varphi$ of $T'$:  We set $\varphi(r):=\phi$ and, for each $v\in V(T')\setminus\{r\}$ we set $\varphi(v):=\varphi'(v)$.  This produces a us-colouring of $T'$.

    For a sufficiently large constant $a$, $a(k-c-1)-r > a(k-c-\log c/\log\log c)$, so each node in $T'$ receives a colour larger than $a(k-c-\log c/\log\log c)$.  Therefore, this partial colouring satisifies requirements (P1) and (P2). In particular, for each $v\in V(T')$,
    \[
       \frac{(\log k)^k}{\gamma(v)^{\gamma(v)}} \le \Gamma(r) \le \frac{(\log k)^k}{\left(\log c+\tfrac{\log c}{\log\log c}\right)^{c+\tfrac{\log c}{\log\log c}}}
   \]
   so $\gamma(v) \ge c+\log c/\log\log c$ and therefore $\varphi(v) \ge a(k-c-\tfrac{\log c}{\log\log c})+1 \ge a(k-\gamma(v)-1)+1$, as required by (P2).

   Now, $T-T'$ is a forest of trees $T_1,\ldots,T_d$ rooted at nodes $r_1,\ldots,r_d$, respectively.  For each $i\in\{1,\ldots,d\}$ and each $v\in V(T_i)$,
   \[
       \Gamma(r_i) \le \frac{(\log k)^k}{\left(\log\left(c+1+\tfrac{\log (c+1)}{\log\log (c+1)}\right)\right)^{c+1+\tfrac{\log (c+1)}{\log\log(c+1)}}} \enspace .
   \]
   Therefore, we can apply induction on each tree $T_i$ with the value $c'=c+1$ and using the colour
   $\phi':=a(k-c)-r-1$.  This choice of $\phi'$ ensures that
   \[
       \phi' \ge a\left(k-c-\tfrac{\log c}{\log\log c}\right) >  a\left(k-c'\tfrac{\log c'}{\log\log c'}\right)
   \]
   so that requirements \cref{total-weight-i} and \cref{phi-bound} are satisified with $c=c'$ and $\phi=\phi'$.
\end{proof}



\begin{lem}\label{tree-ii}
Let $c,k$ be integers $0\le c< k$, let $T$ be a non-empty tree with root $r$, let $\gamma:V(T)\to\N$ be an assignment of positive integer weights to the vertices of $T$.  Then there exists a constant $a>0$ such that,
if
\begin{equation}
   \Gamma(r):=\sum_{v\in V(T)} \frac{(\log k)^k}{(\log \gamma(v))^{\gamma(v)}} \le \frac{(\log k)^k}{(\log c)^c} \enspace . \label{total-weight-ii}
\end{equation}
and
\begin{equation}
    \phi \in \{a(k-c-1)+1,\ldots,ak\} \label{phi-bound-ii}
\end{equation}
then  $T$ has a humped us-colouring $\varphi:V(T)\to\{1,\ldots,ak\}$ such that
\begin{compactenum}[(P1)]
    \item $\varphi(r)=\phi$;
    \item for each $i\in\{a(k-c-1)+1,\ldots,ak\}$, there is at most one $v\in V(T)$ with $\varphi(v)=i$.
    \item for each $v\in V(T)$, $\varphi(v)\ge a\cdot(k-\gamma(v)-1)+1$.
\end{compactenum}
\end{lem}

\begin{proof}
    The proof is by induction on $|V(T)|$. In the base case, $|V(T)|=1$, there is no choice but to set $\varphi(r)=\phi$.  This certainly satisifed (P1) and (P2).  Furthermore, \cref{total-weight-ii} implies that $\gamma(r)\ge c$.  Therefore \cref{phi-bound-ii} ensures that (P3) is satisfied.

    Now we may assume that $|V(T)|\ge 2$.  For each node $v$ of $T$, let $\Gamma(v):=\sum_{u\in V(T_v)}(\log k)^k/(\log(\gamma(u)))^{\gamma(u)}$ where $T_v$ denotes the subtree of $T$ induced by $v$ and all its descendants.  We say that a node $v$ of $T$ is \emph{heavy} if
    \[
        \Gamma(v)> \frac{(\log k)^k}{\log\left(c+\frac{\log c}{\log\log c}\right)^{c+\frac{\log c}{\log\log c}}} \enspace .
    \]
    By fighting a little with this equation we can show that, for any heavy node $v$,
    \begin{equation}
        \Gamma(v) > \left(\frac{\alpha}{c}\right)\cdot
            \frac{(\log k)^k}{(\log c)^c} \enspace . \label{heavy-size}
    \end{equation}

    Let $T'$ be the subtree of $T$ induced by its heavy nodes.  We say that a node of $T'$ is \emph{special} if it is $v$, or it is a leaf or it has at least two children and we let $S$ denote the set of special nodes.  Together, \cref{heavy-size} and \cref{total-weight-ii} imply that the number of leaves of $T'$ is at most $c/\alpha$.  In any tree with $\ell$ leaves, the number of nodes with at least two children is at most $\ell-1$.  Therefore the number of special nodes of $T'$ is at most $2c/\alpha$.

    Therefore, for $a\ge 2/\alpha$, one can obtain a humped us-colouring of $T'$ as follows (this is similar to the proof of \cref{skeleton-colour}):  First colour $r$ with $\varphi(r):=\phi$.  Next, assign each special node $v\in S\setminus\{r\}$ $v$ a distinct colour $\varphi(v)\in\{a(k-c-1)+1,\ldots,ak\}$.  Finally, us-colour the graph $T'-S$ (a collection of paths) using colours $\{a(k-c-1)-2,a(k-c-1)-1,a(k-c-1)\}$.

    This gives a colouring $\varphi$ of $T'$ that certainly satisfies (P1) and (P2).  To see that it also satisfies (P3), observe that, since $|V(T)|\ge 2$, \cref{total-weight-ii} implies that $\gamma(v) > c$ for all $v\in V(T)$. Since $\gamma$ is integer valued, this means that $\gamma(v)\ge c+1$.  Therefore, for each $v\in V(T')$, $\varphi(v)\ge a(k-c-1)-2 \ge a(k-c-2)+1\ge a(k-\gamma(v)-1)+1$ so that $\varphi$ satisfies (P3).

    We are now left with the problem of extending the colouring $\varphi$ of $T'$ to a colouring of $T$.  The forest $T-T'$ contains trees $T_1,\ldots,T_m$ with roots $r_1,\ldots,r_m$, respectively.  For each $i\in\{1,\ldots,m\}$, $r_i$ is a light node, so
    \[
        \Gamma(r_i) \le \frac{(\log k)^k}{\left(\log c + \frac{\log c}{\log\log c}\right)^{c+\frac{\log c}{\log\log c}}}
    \]
    Now can now apply \cref{tree-slack} with the value $\phi = a(k-c-1)-3$ to obtain a us-colouring $\varphi:V(T_i)\to\{1,\ldots,a(k-c-1)\}$ of $T_i$.  It is straightforward to verify that this is indeed a humped us-colouring of $T$.
\end{proof}


For any graph $H$, and any two vertices $v,w\in V(H)$, let $d_H(v,w)$ denote the length of the shortest path, in $H$, from $v$ to $w$. For any $v\in V(H)$ and any $W\subseteq V(H)$, let $d_H(v,W)=\min\{w\in W: d_H(v,w)$. A \emph{BFS layering} of $H$ is a partition $L_0,\ldots,L_m$ of $V(H)$ such that, for each $i\in\{1,\ldots,m\}$ and each $v\in L_i$, $d_H(v,L_0)=i$.

When a BFS layering $L_0,\ldots,L_m$ of $H$ is given, we use the notations $H_i=H[L_i]$, and $H_{\ge i}=H[\bigcup_{j=i}^m L_i]$.  When $H$ is a $t$-tree and $L_0=\{v_1,\ldots,v_t\}$ is a $t$-clique in $H$, each component of $H_i$ is a $(t-1)$-tree.  It is helpful to think of $H_{\ge i}$ as a union of overlapping $t$-trees, each of which is rooted at a $t$-clique in $H_i$. In this way, $H_i$ is a $(t-1)$-tree with $t$-trees hanging off its cliques.

For each $t$-clique $C$ in $H_i$, we use $H_C$ to denote the component of $H[C\cup\bigcup_{j=i+1}^m L_j]$ that contains $C$.  If $v_1,\ldots,v_n$ is a construction order for $H$ then for any component $H'$ of $L_i$, the subsequence $v_{i_1},\ldots,v_{i_|H'|}$ of $v_1,\ldots,v_n$ containing only those vertices in $V(H')$ is a construction order for $H'$.  In this way, each vertex $v_{i_j}$, has a naturally defined parent clique in $H'$.




\begin{lem}\label{two-tree-technical}
    There exists a constant $a>0$ such that, for every integer $k\ge 2$, every $n\le (\log k)^k$, and every $n$-vertex 2-tree $H$, $\uqs(H)\le ak$.
\end{lem}

\begin{proof}
    Let $v_1,\ldots,v_n$ be a construction order for $H$ and
    let $L_0,\ldots,L_m$ be a BFS layering of $H$ beginning at  $L_0=\{v_1,v_2\}$ for some edge $v_2v_2\in E(H)$.  For each edge $vw\in E(H_i)$, recall that $H_{vw}$ is the component of $H[\{v,w\}\cup\bigcup_{j=i+1}^m L_i]$ that contains $vw$.

    For each vertex $v\in L_i$, let
    \begin{equation}
        n_v=\sum_{w\in N_H(v)\cap L_i}|H_{vw}| \enspace . \label{nv}
    \end{equation}

    \begin{clm}\label{size-claim}
        For each $i\in\{1,\ldots,m\}$ and each component $H'$ of $H_{\ge i}$, $\sum_{v\in H'[L_i]} n_v \le 2|H'|$.
    \end{clm}

    \begin{proof}[Proof of \cref{size-claim}]
        Very roughly: the sum double counts each vertex of $H'[L_{i+1}\cup\cdots\cup L_{i+m}]$ exactly twice and double counts the edges of $H'[L_i]$.
        More carefully,
        \begin{align*} |H'|
            & = |H'[L_{i}]| + |H'[L_{i+1}\cup\cdots L_{m}] \\
            & = |H'[L_i]|+\sum_{vw\in E(H'[L_i])} (|H_{vw}|-2) \\
            & = |H'[L_i]|-2|E(H'[L_i])| + \sum_{vw\in E(H'[L_i])} |H_{vw}| \\
            & = |H'[L_i]|-2(|H'[L_i]|-1) + \sum_{vw\in E(H'[L_i])} |H_{vw}| \\
            & = -|H'[L_i]|+2 + \sum_{vw\in E(H'[L_i])} |H_{vw}| \\
            & \ge \sum_{vw\in E(H'[L_i])} |H_{vw}|
        \end{align*}
        On the other hand,
        \begin{align*}
            \sum_{v\in H'[L_i]} n_v
            & = \sum_{v\in H'[L_i]}\sum_{w\in N_H(v)\cap L_i}|H_{vw}| \\
            & = 2 \sum_{vw\in E(H'[L_i])} |H_{vw}| \enspace . \qedhere
        \end{align*}
    \end{proof}

Our proof is based on the following claim:

    \begin{clm}\label{main-claim}
        Let $H'$ be the union of one or more connected component of $H_{\ge i}$.  If $\sum_{v\in V(H')\cap L_i} n_v\le (\log k)^{k-O(1)}/(\log c)^c$ then there exists $\varphi:V(H')\to\{1,\ldots,ak\}$ and $\psi:V(H')\to\{0,1\}$ such that
        \begin{compactenum}[(P1)]
            \item The function $\theta:V(H')\to\{1,\ldots,2ak\}$ defined by $\theta(v)=2\varphi(v) - \psi(v)$ is a humped us-colouring of $H'$;
            \item for each $v\in V(H'[L-i])$, $\psi(v)=0$;
            \item for each $\iota\in\{a(k-c-1)+1,\ldots,ak\}$ there is at most one vertex $v$ in $V(H'[L-i])$ with $\varphi(v)=\iota$.
        \end{compactenum}
    \end{clm}

    \begin{proof}[Proof of \cref{main-claim}]
        In all cases, (P2) requires setting $\psi(v)=0$ for all $v\in V(H'[L_i])$, so we will not mention this again.

        The proof is by induction on $m-i$. In the base case $m-i=0$ and $H'$ is a forest.  In this case, $n_v=1$ for each $v\in V(H')$.  Set $\gamma(v):=k$ and observe that $(\log k)^k/(\log(\gamma(v))^{\gamma(v)}=1=n_v$ for each $v\in V(H')$. Therefore,
        \[
            \sum_{v\in V(H')\cap L_i} \frac{(\log k)^k}{(\log \gamma(v))^{\gamma(v)}} = \sum_{v\in V(H')\cap L_i} n_v
            \le \frac{(\log k)^{k-1}}{(\log c)^c}
            \le \frac{(\log k)^{k}}{(\log c)^c} \enspace .
        \]
        Setting $\phi=k$ and then applying \cref{tree-ii} then gives a colouring $\varphi: V(H')\to\{1,\ldots,ak\}$ that satisfies (P1)--(P3).

        In the general case, we set $\gamma(v):=\max\{c\in\N:(\log k)^k/(\log c)^c \ge n_v\}$ for each $v\in V(H'[L_i])$.  This implies that
        \[
            \frac{(\log k)^k}{(\log(\gamma(v)+1))^{\gamma(v)+1}} > n_v
            \ge \frac{(\log k)^k}{(\log(\gamma(v)))^{\gamma(v)}}  \]
        so
        \[
            \frac{n_v}{(\log k)^{k}} < \log(\gamma(v)+1)^{\gamma(v)+1} \le
            \log(\gamma(v))^{\gamma(v)+O(1)}  \le
            \log(\gamma(v))^{\gamma(v)}\cdot(\log k)^{O(1)}
        \]
        and
        \[
           \sum_{v\in H'[L_i]} \frac{(\log k)^k}{(\log \gamma(v))^{\gamma(v)}} \le \sum_{v\in H'[L_i]} n_v(\log k)^{O(1)} \le \frac{(\log k)^k}{(\log c)^c}
        \]
        Then, by \cref{tree-ii}, we can find a us-colouring $\varphi$ of the tree $H'[L_i]$ that satisfies (P3) and satisfies (P1), at least for the graph $H'[L_i]$. (TODO: Requires that \cref{tree-ii} be stated in terms of a forest.) In addition $\varphi(v)\ge a(k-c-\gamma(v)-1)+1$ for each $v\in V(H'[L_i])$.  It remains to show that we can extend this to a colouring of $V(H')\setminus L_i$.  This extension requires some care.

        For the root $r$ of the tree $H'[L_i]$, we define $D_r=\{x\in L_{i+1}:rx\in E(H)\}$.  For each node $w$ with parent $v$ in $H'[L_i]$, we define $D_w=\{x\in L_{i+1}:wx\in E(H)\}\setminus D_{v}$.  Fix any $w\in V(H'[L_i]$ and consider the graph $H_w:= H'[D_w\cup \bigcup_{j=i+2}^m L_j]$ that contains $w$.  Now, $H_w[L_{i+1}]$ is a rooted forest with roots $w_1,\ldots,w_d$.  We apply the inductive hypothesis to this graph.


        We add a node $w'$ to $H_w$ to obtain a graph on which we can apply induction.  Specifically, $H_w[L_{i+1}]$ is a tree $T_w$ and $H_w$ consists of 2-trees hanging off the edges of $T_w$.




        This induction yields a humped us-colouring of $H'$



        This completes the proof of \cref{main-claim}.
    \end{proof}

    Consider some tree $T'$ in $F_i$. The root $r$ of $T'$ has two parents $a_0$ and $a_1$ in $L_{i-1}$, $a_0a_1$ is an edge of $F_{i-1}$ and, without loss of generality, $a_0\prec a_1$ (so $a_0$ is the parent of $a_1$ in $F_{i-1}$).  We call $a_0$ the \emph{caretaker} of $r$.  We cover the tree $T'$ with two trees $T'_0=T'[\{v\in L_i:va_0\in E(H)]$ and $T'_1=T'[\{v\in L_i:va_1\in E(H)]$.  Note that the only node these two trees have in common is $r$.  The nodes in $V(T'_0)$ are called \emph{tended} nodes.  The nodes in $V(T_1')\setminus \{r\}$ are called \emph{neglected} nodes.

    For a node $a\in L_{i-1}$, let $Q_a:=\{x\in L_{i}:ax\in E(H)\}$.
    If $w$ is a root of $F_i$, then $R_w:=Q_w$.  Otherwise, $w$ has some parent $v\in L_i$, in which case $R_w=Q_w\setminus Q_v$.
    %
    %
    %
    %
    %
    %
    %
    %
    %
    %
    % Turning things around, a node $a_0\in L_{i-1}$ with children $a_1,\ldots,a_d$ in $F_{i-1}$ is the caretaker of one or more roots $r_1,\ldots,r_m$, each of which defines a tree $T'_i$ and $T_i$  in $F_i$.
    %
    %
    %
    %
    %
    % Each non-root in $F_i$ has exactly one parent in $L_{i-1}$ and one parent in $L_i$.
    %
    %
    %
    %
    %
    %
    % Let $L_0,\ldots,L_m$ be a BFS layering of $H$ beginning at some edge $v_1v_2$ (so that $L_0=\{v_1,v_2\}$).  For each $i\in\{0,\ldots,m\}$ and each vertex $v$ in $H[L_i]$, let $H_{v}$ be the component of $H[\{v\}\cup\bigcup_{j=i+1}^m L_i]$ that contains $v$ and let $\gamma(v)$ be the maximum integer such that $(\log k)^k/(\log\gamma(vw))^{\gamma(vw)} \ge |H_{vw}|$.
    %
    % Consider the rooted tree $T:=H[L_i]$.  For each vertex $v\in V(T)$ with children $v_1,\ldots,v_d$, consider the subgraph $H_v$ consisting of the union $H_{vv_1}\cup\cdots H_{vv_d}$.
    % For each edge $vw$ in $H[L_i]$
    %
    %
    %
    %   apply induction on the graph labelled

\end{proof}

\cref{two-tree-technical} gives our main result for 2-trees:

\begin{thm}
    For any integer $k>1$, any $n\le \tfrac{1}{2}(\log k)^k$, and any $n$-vertex 2-tree $H$, $\uqs(H)\le k = O(\log n/\log^{(3)} n)$.
\end{thm}

\begin{proof}
    Apply \cref{two-tree-technical} with $i=0$ and $c=2$.  This shows that $\hus(H)\le k$ and requires that $n = |H_{a_1a_2}| \le (\log k)^k$, which is satisfied for $k \in O(\log n/\log^{(3)} n)$.  Applying \cref{humped} then shows that $\uqs(H)\le\hus(H) \in O(\log n/\log^{(3)} n)$.
\end{proof}
















\subsection{The Lower Bound for $t$-Trees}

\begin{lem}\label{apex-graph}
    Let $h,k\ge 1$ be integers, let $U$ be a graph with $\uqs(U)\ge h$ and let $G$ be a graph obtained by taking $k+1$ disjoint copies $U_0,\ldots,U_k$ of $U$ and adding an apex vertex $a$ adjacent to each $v\in\bigcup_{i=0}^k V(U_i)$.  Then, for any integer $k_0\in \{1,\ldots,k\}$ and any us-colouring of $\varphi:V(G)\to\{k_0,\ldots,k\}$, $\varphi(a) \ge k_0+1$.
\end{lem}

\begin{proof}
    Since $\uqs(U_i)\ge h$, there exists $v_i\in V(U_i)$ such that $\varphi(v_i)\ge h$, for each $i\in\{0,\ldots,k\}$.  Since $|\{0,\ldots,k\}=k+1>k-k_0+1=|\{k_0,\ldots,k\}|$ the Pigeonhole Principle implies that there exists distinct $i,j\in\{0,\ldots,k\}$ such that $\varphi(v_i)=\varphi(v_j)$.  Since $v_i a v_j$ is a path in $G$, this implies that $\varphi(a)\ge \varphi(v_i)+1\ge h+1$.
\end{proof}

For a graph $U$ with and integers $h,\ell\ge 0$, we define the $(h,\ell)$-boost $U^{(h,\ell)}$ of $U$ as follows: The vertex set of $U^{(h,\ell)}$ is the disjoint union of $L_0,\ldots,L_\ell$.  The set $L_0:=\{a_0\}$ consists of a single vertex. For each $i\in\{1,\ldots,\ell\}$ and each $a\in L_{i-1}$, $U^{(h,\ell)}$ contains $h\ell+1$ disjoint copies $U_{a,0},\ldots,U_{a,h\ell}$ of $U$ and contains the edge $av$ for each $v\in\bigcup_{j=0}^{h\ell} V(U_j)$.  This determines the set $L_i=\bigcup_{a\in L_{i-1}}\bigcup_{j=0}^{h\ell} V(U_{a,j})$.

\begin{lem}\label{boost}
    For any graph $U$, any integer $\ell\ge 0$, and any $h\ge\uqs(U)$, $\uqs(U^{(h,\ell)})\ge h\ell +1$.
\end{lem}

\begin{proof}
    Suppose, for the sake of contradiction, that $\uqs(U^{(h,\ell)})=k<h\ell+1$ and let $\varphi:V(U^{(h,\ell)})\to\{1,\ldots,k\}$ be a us-colouring of $U_{(h,\ell)}$.  Let $L_0,\ldots,L_{\ell}$ be the partition of $V(U^{(h,\ell)})$ used in the definition of $V(U^{(h,\ell)})$.
    % (Alternatively, $L_0,\ldots,L_{\ell}$ are the breadth-first-search layers of $U^{(h,\ell)}$ rooted at $a_0$.)
    We will show by induction on $\ell-i$ that, for each $a\in L_{i}$, $\varphi(a)\ge(\ell-i)h+1$. This gives the desired contradiction since it implies that, for the unique vertex $a_0\in L_0$, $\varphi(a_0)\ge \ell h+1 > k$.

    The base case of the induction, $\ell-i=0$, is trivial; it simply asserts that $\varphi(v)\ge 1$ for each $v\in L_\ell$.  For any $i\in\{0,\ldots,\ell-1\}$ we apply the inductive hypothesis to conclude that $\varphi(v)\in\{(\ell-i-1)h+1,\ldots,k\}$ for each $v\in L_{i+1}$.  For each $a\in L_i$, the subgraph of $U^{(h,\ell)}$ induced by $a$ and its neighbours in $L_{i+1}$ contains the graph described in \cref{apex-graph} with $k_0:=(\ell-i-1)h+1$.  The conclusion of \cref{apex-graph} therefore implies that $\varphi(a)\ge k_0+h=(\ell-i)h+1$, as required.
\end{proof}

\begin{lem}\label{boost-size}
    For any graph $U$ and any integers $h,\ell \ge 1$, $|U^{(h,\ell)}| \le (|U|(h\ell))^{\ell}\cdot (1+O((|U|h\ell)^{-1})$.
\end{lem}

\begin{proof}
    It is easy to see that, for each $i\in \{0,\ldots,\ell\}$, $|L_i|=(|U|(h\ell+1))^i$.  Therefore,
    \[ |U^{h,\ell}| = \sum_{i=0}^\ell |L_i| = \sum_{i=0}^\ell (|U|(h\ell+1))^i = (|U|(h\ell))^{\ell}\cdot (1+O((|U|h\ell)^{-1}) \enspace . \qedhere
    \]
\end{proof}


\begin{lem}\label{boost-treewidth}
    For any graph $U$ and any integers $h,\ell\ge$, $\tw(U^{(h,\ell)})\le \tw(U)+1$.
\end{lem}

\begin{proof}
  Create a width-$(\tw(U)+1)$ tree-decomposition $(B_x:x\in V(T))$ of $U^{(h,\ell)}$ as follows: Start with $T$ having a single node $z_0$ with $B_{z_0}=L_0$.  For each $i\in\{1,\ldots,\ell\}$, and each $a\in L_{i-1}$, find some bag $B_z$ in the current decomposition that contains $a$, take $h+1$ disjoint copies $(A_x:x\in V(T_0)),\ldots,(A_x:x\in V(T_h))$ of some width-$t$ tree decomposition $\mathcal{T}$ of $U$.  For each $i\in\{0,\ldots,h\}$, add an edge from $z$ to any node of the tree in $T_i$ and add $a$ to every bag in $T_i$.  It is straightforward to verify that this does, indeed, give a width-$\tw(U)+1$ tree-decomposition of $U^{(h,\ell)}$.
\end{proof}


\begin{thm}\label{treewidth-lower-bound}
    For each integer $t\ge 0$, there exists $\alpha,\beta>0$
    such that, for each integer $\ell\ge 2^{2^{2^{\ddots}}}$, there exists a graph $G$, with $|G|\le (\log^{(t-1)}\ell)^{\alpha\ell}$, $\tw(G)\le t$, and $\uqs(G)\ge (\beta\ell\log\ell)/\log^{(t)}(\ell\log\ell)$.
\end{thm}

\begin{proof}
    The case $t=0$ is trivial; For every $n\ge 1$, taking $G$ to be an $\ell^\ell$-vertex graph with no edges gives a graph with $\tw(G)=0$ and
    \[
        \uqs(G)=1=\frac{\ell\log \ell}{\ell\log\ell} = \frac{(\ell\log \ell)}{\log^{(0)}(\ell\log\ell)} \enspace .
    \]
    This establishes the result for $t=0$ with $\alpha=\beta=1$.

    The remainder of the proof is by induction on $t$.  The base case $t=1$ has already been established by \citet{karpas.neiman.ea:on} who show that the complete $(\ell+1)$-ary tree $T$ of height $\ell-1$ has $\uqs(T)\ge \ell$.  The tree $T$ has size $\sum_{i=0}^{\ell-1} (\ell+1)^i \le \ell^\ell$.
    Observe that
    \[
        \uqs(T) \ge \ell = \frac{\ell\log\ell}{\log\ell} \ge \frac{\ell\log\ell}{\log(\ell\log\ell)} = \frac{\ell \log \ell}{\log^{(1)}(\ell\log\ell)}
    \]
    This establishes the result for $t=1$ with the constant $\alpha=\beta=1$.

    For $t>1$ we can apply the inductive hypothesis to obtain a graph $U$, with $\tw(U)\le t-1$, $|U|=(\log^{(t-2)}m)^{\alpha'm}$ and $\uqs(U)\ge \beta' m\log m/\log^{(t-1)}(m\log m)=:h$.
    We do this with the value $m:=\log^{(2)}(\ell^\ell)/\log^{(t+1)}(\ell^\ell)=\log (\ell\log\ell)/(\log^{(t)}(\ell\log\ell))$.  Now we take the graph $G:=U^{(h,\ell)}$.  By \cref{boost-treewidth}, $\tw(U^{(h,\ell)})\le \tw(U)+1\le t$.  By \cref{boost},
    \begin{align*}
       \uqs(G) & \ge \ell h+1 > \ell h\\
               & = \frac{\ell \beta' m\log m}{\log^{(t-1)}(m\log m)} \\
               & = \left(\beta'\ell m\right)\left(\frac{\log m}{\log^{(t-1)}(m\log m)}\right) \\
               & = \left(\frac{\beta'\ell \log(\ell\log\ell)}{\log^{(t)}(\ell\log\ell)}\right)\left(\frac{\log m}{\log^{(t-1)}(m\log m)}\right) \\
               & \ge \left(\frac{\beta'\ell \log\ell}{\log^{(t)}(\ell\log\ell)}\right) \left(\frac{\log m}{\log^{(t-1)}(m\log m)}\right) \\
               & \ge \frac{(\beta'/2)\ell \log\ell}{\log^{(t)}(\ell\log\ell)}
   \end{align*}
   where the final inequality holds for all $t\ge 2$ and $m\ge 1$ (which holds for all $\ell\ge 2$).

   By \cref{boost-size}, the size of $G$ is given by
   \begin{align*}
        |G| & \le \gamma\cdot \left((\log^{(t-2)}m)^{\alpha' m}\ell h\right)^\ell \\
        & = \gamma ((\log^{(t-2)}m)^{\alpha'\ell m}) (\ell^\ell) (h^\ell) \\
        & = \gamma \left(\log^{(t-2)}\left(\frac{\log(\ell\log\ell)}{\log^{(t)}(\ell\log\ell)}\right)\right)
        ^{\frac{\alpha'\ell\log(\ell\log\ell)}{\log^{(t)}(\ell\log\ell)}} \cdot (\ell^\ell) (h^\ell) \\
        & \le \gamma \left(\log^{(t-2)}(\log(\ell\log\ell))\right)
        ^{\frac{\alpha'\ell\log(\ell\log\ell)}{\log^{(t)}(\ell\log\ell)}} \cdot (\ell^\ell) (h^\ell) \\
        & = \gamma \left(\log^{(t-1)}(\ell\log\ell)\right)
        ^{\frac{\alpha'\ell\log(\ell\log\ell)}{\log^{(t)}(\ell\log\ell)}} \cdot (\ell^\ell) (h^\ell) \\
        & = \gamma (\ell\log\ell)^{\alpha'\ell} (\ell^\ell) (h^\ell) \\
        & = \ell^{(\alpha'+1)\ell + \frac{\ell\log h + \ell\log\log\ell \log\gamma}{\log\ell}} \\
        & = \ell^{(\alpha'+4)\ell}
   \end{align*}
   for all sufficiently large $\ell$.
\end{proof}

Rewriting \cref{treewidth-lower-bound} in terms of $n:=|G|$, we obtain a more-readily digestible corollary:

\begin{cor}\label{treewidth-lower-bound-n}
    For every integer $t\ge 0$, there exists a constant $\alpha>0$ such that, for infinitely many $n\in\N$,   there exists an $n$-vertex graph $G$ with $tw(G)\le t$ and  $\uqs(G)\ge \alpha\log n/\log^{(t+1)} n$.
\end{cor}

Since every graph of treewidth at most 2 is planar, we obtain the following corollary from \cref{treewidth-lower-bound-n}:

\begin{cor}\label{planar-lower-bound}
    There exists a constant $\alpha>0$ such that, for infinitely many $n\in\N$, there is an $n$-vertex planar graph $G$ with $\uqs(G)\ge \alpha\log n/\log^{(3)} n$.
\end{cor}


% \section{Lower Bound for 2-trees}
%
% Let $T_{b,h}$ denote a complete $b$-ary rooted tree of height $h$ (the rooted tree in which every leaf has depth $h$ and every non-leaf has exactly $b$ children).  The following lemma is due to \citet{karpas.neiman.ea:on}. We include a proof here for the sake of completeness.
%
% \begin{lem}\label{b-h-tree}
%     Let $T=T_{b,h}$ for some $b\ge 1$, $h\ge 0$, let $r$ be the root of $T$, and let $\varphi:V(T)\to \{1,\ldots,b-1\}$ be a us-colouring of $T_{b,h}$.  Then $\varphi(r)\ge h+1$.
% \end{lem}
%
% \begin{proof}
%     The proof is by induction on $h$. The base case $h=0$ is trivial: It simply states that $\varphi(r)\ge 1$.
%
%     For $h>0$, apply the inductive hypothesis to each of the maximal subtrees rooted at the $b$ children $r_1,\ldots,r_{b+1}$ of $r$ to determine $\varphi(r_i)\in\{h,\ldots,b-1\}$.  Since $|\{h,\ldots,b-1\}|=b-h< b$, there are two distinct $i,j\in\{1,\ldots,b\}$ such that $\varphi(r_i)=\varphi(r_j)$.  Since $r_irr_j$ is a path in $T$ and $\varphi$ is a us-colouring, this implies that $\varphi(r)\ge \varphi(r_i)+1\ge h+1$.
% \end{proof}
%
% \begin{lem}\label{apex-graph}
%     Let $A$ be the graph consisting of $k$ disjoint copies of $T_{h+1,h}$ plus an apex vertex $a$, and let $\varphi:V(A)\to \{1,\ldots,k\}$ be a us-colouring of $A$.  Then $\varphi(a)\ge h+2$.
% \end{lem}
%
% \begin{proof}
%     Let $T_1,\ldots,T_{k}$ denote the copies of $T_{h+1,h}$ in $A$.
%     By \cref{b-h-tree} any us-colouring of $T_i$ uses at least $b=h+1$ distinct colours or uses a colour greater than or equal to $h+1$.  In either case, $T_i$ contains a node $v_i$ with $\varphi(v_i)\in \{h+1,\ldots,k\}$, for each $i\in\{1,\ldots,k-h+1\}$.  Since $|\{h+1,\ldots,k\}|=k-h<k$, this implies that there exists distinct $i,j\in\{1,\ldots,k-h+1\}$ with $\varphi(v_i)=\varphi(v_j)$.  Since $v_iav_j$ is a path in $A$, this implies that $\varphi(a)\ge \varphi(v_i)+1\ge h+2$.
% \end{proof}
%
% Define a graph $G_{\ell,h}$ whose vertex set is the disjoint untion of $L_0,\ldots,L_\ell$, defined as follows.  $L_0=\{r\}$ is a single vertex.  For each $i\in\{1,\ldots,\ell\}$ and each vertex $a\in L_{i-1}$, create $k$ copies of $T_{h+1,h}$ that are dominated by $a$.  Add all of these newly created vertices to $L_i$.
%
% \begin{lem}
%     For any us-colouring $\varphi:V(G_{\ell,h})\to\{1,\ldots,k\}$ of $G_{\ell,h}$, any $i\in\{0,\ldots,\ell\}$ and any $v\in L_i$, $\varphi(v)\ge (\ell-i)(h+1) + 1$.
% \end{lem}
%
% \begin{proof}
%     The proof is by induction on $\ell-i$.  The base case $i=\ell$ (so $\ell-i=0$) is trivial, it simply states that node $v\in L_\ell$ receives a colour $\varphi(v)\ge 1$.
%
%     For $i < \ell$, consider the set $S_v$ consisting of neighbours of $v$ in $L_{i+1}$.  By induction, each $w\in S_v$ receives a colour $\varphi(w)\in \{(\ell-i-1)(h+1)+1,\ldots,k\}$.  Now applying \cref{apex-graph} to the induced graph $G_{\ell,h}[\{v\}\cup S_v]$ shows that $\varphi(v)\ge (\ell-i-1)(h+1)+h+2=(\ell-i)(h+1)+1$.
% \end{proof}
%
%
% \begin{thm}
%     For infinitely many values of $n$ there exists an $n$-vertex $2$-tree $G$ such that $\uqs(G)\in \Omega(\log n)$.
% \end{thm}
%
% \begin{proof}
%     Use the graph $G_{\ell,h}$ with $h=\alpha\log\log n-1$, and $\ell=\beta\log n/\log\log n$.  The graph has size
%     \[  (k(h+1)^{h+1})^\ell = n^{O(\alpha+\beta)} \enspace  \]
%     and requires $\ell(h+1) = \Omega(\alpha\beta\log n)$ colours.
% \end{proof}

\section{Colouring Planar Graphs}

\begin{lem}
    For any two graphs $G_1$ and $G_2$, $\uqs(G_1\boxtimes G_2)\le \uqs(G_1)\cdot\uqs(G_2)$.
\end{lem}

\begin{proof}
  Use a product colouring $\varphi(x,y)=\uqs(G_2)\cdot \rho(x) + \psi(y)$ where $\rho:V(G_1)\to\{0,\ldots,\uqs(G_1)-1\}$ and $\psi:V(G_2)\to\{0,\ldots,\uqs(G_2)-1\}$ are us-colourings of $G_1$ and $G_2$.  Consider any path $uvw$ in $G_1\boxtimes G_2$.  If the projection of $uvw$ onto $G_1$ is a path, then contribution of $\psi$ to $\varphi$ is irrelevant and the fact that $\rho$ is a us-colouring of $G_1$ implies that $\varphi(u)\neq \varphi(w)$ or that $\varphi(v)>\varphi(w)$.  If the projection of $uvw$ onto $G_1$ has only two vertices, then $uvw$ is a $3$-cycle in $G_1\boxtimes G_2$ and $\varphi(u)$, $\varphi(v)$ and $\varphi(w)$ are all distinct.  If the projection of $uvw$ onto $G_1$ has only one vertex then the projection of $uvw$ onto $G_2$ has three vertices.  In this case, the contribution of $\rho$ to $\varphi$ is irrelevant and  and the fact that $\rho$ is a us-colouring of $G_2$ implies that $\varphi(u)\neq \varphi(w)$ or that $\varphi(v)>\varphi(w)$.
\end{proof}

\section{Other Ideas}

Approximation algorithm for trees.  Using characterization in terms of $T_{h+1,h}$.

Extension to $\chi_\ell(G)$ for constant values of $\ell$.















%
%
%
%
%
% A \emph{factorial tree} $F_h$ of height $h$ is a rooted tree in which, for each $d\in\{0,\ldots,h-1\}$, each node of depth $d$ has $d+2$ children and each node of depth $h$ is a leaf.  A \emph{partial factorial tree} $F_{d,h}$ is a maximal subtree of $F_h$ rooted at a node of depth $d$. The following lower bound is due to \citet{karpas.ea:}.  We present a proof here for the sake of completeness.
%
%
% \begin{lem}\label{factorial-tree}
%     Let $T=F_{d,h}$ be a partial factorial tree rooted at some node $r$.
%     For any us-colouring $\varphi:V(T_{d,h})\to\{1,\ldots,h\}$,
%     $\varphi(r)\ge h-d+1$.
% \end{lem}
%
% \begin{proof}
%     The proof is by induction on $h-d$.    The base case $h-d=0$ is trivial: It simply states that $\varphi(r)\ge 1$.
%
%     For the inductive step, we apply the inductive hypothesis to each of the subtrees rooted at the children $r_1,\ldots,r_{d+2}$ of $r$ to determine that $\varphi(r_i)\in \{h-d,\ldots,h\}$ for each $i\in\{1,\ldots,d+2\}$.  Since $|\{h-d,\ldots,h\}|=d+1 < d+2$, $\varphi(r_i)=\varphi(r_j)\ge h-d$ for some distinct pair $i,j\{1,\ldots,d+2\}$.  Therefore $\varphi(r) \ge \varphi(r_i)+1 = h-d+1$.
% \end{proof}
%
% By applying \cref{factorial-tree} with $d=0$, we obtain
% \begin{cor}\label{factorial-tree-2}
%     Any us-colouring $\varphi:V(F_h)\to\N$ of $F_h$ uses at least $h+1$ distinct colours, i.e, $|\{\varphi(v):v\in V(F_h)\}|\ge h+1$.
% \end{cor}
%
%
% \begin{lem}\label{tree-plus-apex}
%     Let $k,h\in\N$ with $k > h+1$, let $G$ be the graph consisting of $k-h+1$ disjoint copies of $F_h$ plus an apex vertex $r_0$, and let $\varphi:V(G)\to\{1,\ldots,k\}$ be a us-colouring of $G$.  Then $\varphi(r_0) \ge h+2$.
% \end{lem}
%
% \begin{proof}
%     Let $T_1,\ldots,T_{k-h+1}$ denote the copies of $F_h$ in $G$.
%     By \cref{factorial-tree-2}, each copy of $T_i$ requires at least $h+1$ distinct colours and therefore $\varphi(v_i)\in\{h+1,\ldots,k\}$ for at least one node $v_i\in V(T_i)$. Since $|\{h+1,\ldots,k\}|=k-h< k-h+1$ there exists distinct $i,j\in\{1,\ldots,k-h+1\}$ such that $\varphi(v_i)=\varphi(v_j)\ge h+1$.  Since $v_ir_0v_j$ is a path in $G$, this implies that $\varphi(r_0)\ge \varphi(r_i)+1 \ge h+2$.
% \end{proof}
%
% Let $k=\ell(h+1)$ be a multiple of $h+1$ and define a graph $G_{k,h}$ as follows: $G_{k,h}$ has a set $L_{\ell}=\{r\}$ consisting of exactly one node.
% For each $i\in\{0,\ldots,\ell-1\}$, and each vertex $v\in L_{i+1}$, we create $k-(\ell-i)(h+1)$ disjoint copies of $F_h$ that are each dominated by $v$. These newly created vertices are added to the set $L_i$.
%
% \begin{lem}
%     For any us-colouring $\varphi:V(G_{k,h})\to \{1,\ldots,k\}$, any $i\in\{1,\ldots,\ell\}$, and any $v\in L_i$, $\varphi(v)\ge i(h+1)+1$.
% \end{lem}
%
% \begin{proof}
%     The proof is by induction on $\ell$.  In the base case $\ell=1$, $v\in L_1$ and its neighbours in $L_2$ induce the graph $G$ described in \cref{tree-plus-apex} and
% \end{proof}
%
%
% }
% \begin{clm}
%   For each $i\in\{0,\ldots,\ell\}$, $|S_i|=i!((h+1)(h+1)!)^i$.
% \end{clm}
%
% \begin{proof}
%   The proof is by induction on $i$.  By definition, $|S_0|=1=0!((h+1)(h+1)!)^0$, so this establishes the base case $i=0$. For each $i\in\{1,\ldots,\ell\}$,
%   \[
%      |S_i|=|S_{i-1}|\cdot i(h+1)(h+1)! = i!((h+1)(h+1)!)^i.
%   \]
% \end{proof}
%
% \begin{clm}
%     $|V(G_{k,h})| \le \ell^\ell (h+1)^\ell (h+1)^k
%     (1+o_h(1))\ell!((h+1)(h+1)!)
%
% \end{clm}
%
%
%
%
%
%
%
%
% For $h,\ell\in \N$, let $G_{\ell,h}$ be a rooted graph defined as follows: $G_{1,h}$ is the graph $G$ described in \cref{tree-plus-apex} rooted at the apex vertex $r_0$.  For $\ell>1$, $G_{\ell,h}$ is obtained by taking $(h+1)!$ copies of $G_{\ell-1,h}$ with apex vertices $r_1,\ldots,r_{(h+1)!}$, taking one copy of $G_{1,h}$ with non-apex vertices $v_1,\ldots,v_{(h+1)!}$
%
%
%
% \begin{lem}
%     Let $G$ be
% \end{lem}



% \begin{lem}
%     Any humped us-colouring of $T_{a,a}$ requires at least $a$ distinct colours.
% \end{lem}
%
% \begin{lem}
%     Let $G:= (k-a)\times T_{a,a}$ consist of $k-a$ disjoint copies of $T_{a,a}$.
% \end{lem}

% \section{$2$-Trees}
%
% In this section we consider how to solve this problem for 2-trees.
%
% Let $T_{b,h}$ denote a complete $b$-ary tree of height $h$ (the rooted tree in which every leaf has depth $h$ and every non-leaf has exactly $b$ children).
%
% \begin{lem}\label{big-complete-tree}
%     Let $k\ge x\ge 1$ be integers.
%     Let $T$ be an $r$-rooted tree that contains no copy of $T_{k,k}$ and such that $T$ contains no copy of $T_{k,h}$ rooted at $r$.  Then, for each $c\in\{h,\ldots,2k\}$, $T$ has a us-colouring $\varphi:V(T)\to \{1,\ldots,2k\}$ such that $\varphi(r)=c$.
% \end{lem}
%
% \begin{proof}
%     The proof is by induction on $|V(T)|$.  The base case $|V(T)|=1$ is trivial.
%
%     For $|V(T)|>1$, let $r_1,\ldots,r_d$ be the children of $r$ and, for each $i\in\{1,\ldots,d\}$, let $T_i$ be the subtree of $T$ rooted at $r_i$.  The subtrees $T_1,ldots,T_d$ can be partitioned into two sets $T_1,\ldots,T_{d'}$ and $T_{d'+1},\ldots,T_d$ so that $T_i$ contains a copy of $T_{k,h-1}$ that includes $r_i$ if and only if $i\le d'$.  Note that, possibly, $d'=0$.   Certainly, $d' < k$ since, otherwise $T$ contains a copy of $T_{k,h}$ rooted at $r$.
%
%     Now, set $\varphi(r):=c$. For each $i\in\{1,\ldots,d'\}$, set $\varphi(r_i)$ to a distinct value in $\{k+1,\ldots,2k\}\setminus \{c\}$. For each $i\in\{d'+1,\ldots,d\}$, set $\varphi(r_i):=h-1 < c$.  By induction on each tree $T_i$, $\varphi$ can be completed to a humped us-colouring of $T_i$.  It is easy to verify that, such a colouring gives a humped us-colouring of $T$.
% \end{proof}
%
% A colouring $\varphi:V(G)\to\N$ is a \emph{$(>r)$-unique} if there is no colour $r'>r$ such that $\varphi(v)=\varphi(w)=r'$ for two distint vertices $v,w\in V(G)$.
%
% For a tree $T$ and integer $k\ge \hus(T)$, let $\hus(T,k)$ denote the smallest integer $r$ such that $T$ has a humped $(>r)$-unique humped us-colouring $\varphi:V(G)\to\{1,\ldots,k\}$.
%
% \begin{lem}\label{big-repeats}
%     Let $k$ and $x$ be integers with $k\ge 2x+1$.
%     If $\hus(T,k)> 2x+1$, then $|V(T)|\ge (k-2x+2)\cdot x^x$.
% \end{lem}
%
% \begin{proof}
%     Let $T_1:= T$, let $i:=1$ and while $T_i$ contains a copy of $T_{x,x}$,
%     select any maximum-depth node $r_i\in V(T_i)$ such that the subtree $T_{r_i}$ of $T_i$ rooted at $r_i$ contains a copy of $T_{x,x}$.  Set $T_{i+1}=T_i-T_{r_i}$ and set $i:=i+1$.  Note that choosing each $r_i$ to be of maximal depth ensures that $T_{r_i}$ does not contain a copy of $T_{x+1,x+1}$.
%
%     This yields a sequence of trees $T_1,\ldots,T_m$ where, for each $i\in\{1,\ldots,m-1\}$, $|V(T_i)|\ge x^x$ and, by \cref{big-complete-tree} $\hus(T_i) \le 2x+1$.  We claim that $m-1 > k-2x$, so $m\ge k-2x+2$ and therefore $|V(T)|\ge (k-2x+2)\cdot x^x$, thus completing the proof.
%
%     To see why this claim is true observe that, otherwise, we could obtain a $(>2x+1)$-unique humped us-colouring $\varphi:V(T)\to\{1,\ldots,k\}$ as follows: For each $i\in\{1,\ldots,m-1\}$ take a humped us-colouring of $T_i$ using $\{1,\ldots,2x+1\}$ so that $\varphi(r_i)=2x+1$.  Take a humped us-colouring of $T_m$ using $\{1,\ldots,2x\}$.   Finally, for each $i\in\{1,\ldots,m-1\}$, replace $\varphi(r_i)$ with a distinct colour in $\{2x+1,\ldots,k\}$.  This certainly produces a $(>2x+1)$-unique colouring and it is easy to verify that this colouring is indeed a humped us-colouring.
% \end{proof}
%
% For a function $f:V(G)\to\R^+$, a colouring $\varphi:V(G)\to\N$ is $f$-lower-bounded if $\varphi(v)\ge f(v)$ for each $v\in V(G)$.  For a tree $T$ and $f:V(G)\to\N$, we let $T^f$ denote the tree obtained by replacing each leaf $v$ of $T$ with $T_{\lceil f(v)/2\rceil,\lceil f(v)/2\rceil
% }$.  \cref{big-complete-tree,big-repeats} generalize naturally to the following.
%
% \begin{lem}\label{big-complete-tree}
%     Let $k\ge x\ge 1$ be integers, let $T$ be a tree and let $f:V(T)\to\R$.  Then $T$ has an $f$-lower bounded humped us-colouring $\varphi:V(T)\to\{1,\ldots,2k\}$ or $T$ contains a subtree $S$ such that $S^f$ contains $T_{k,k}$.
% \end{lem}
%
% \begin{lem}\label{big-repeats}
%     Let $k$ and $x$ be integers with $k\ge 2x+1$.
%     If $\hus(T,k)> 2x+1$, then $|V(T)|\ge (k-2x+2)\cdot x^x$.
% \end{lem}
%
%
%
%
% a \emph{weight-constrained}
%
%
%
%
% \section{Generalization to $t$-Trees}
% \seclabel{t-trees}
%
%
% For the remainder of this section, $H$ is a $t$-tree on $n\ge t+1$ vertices, $\pi_1,\ldots,\pi_n$ is a construction order for $H$, and $\mathcal{T}:=(B_x:x\in V(T))$ is a canonical tree decomposition of $H$.  It is easy to see that every bag of $\mathcal{T}$ has size exactly $t+1$, $T$ is rooted at $r$, and
% \begin{inparaenum}[(i)]\setcounter{enumi}{2}
%   \item for each node $y$ of $T$ with parent $x$, $|B_y\setminus B_x|=1$.\label{three}
% \end{inparaenum}
% Note that this implies that $|V(T)|=n-t$, since $B_r$ contains $t+1$ vertices of $H$ and, by (\ref{three}), each $x\in V(T)\setminus\{r\}$ includes exactly one vertex of $H$ that is not in its parent.
%
% For each $v\in V(T)$ let $x_T(v)$ be the minimum-depth node $x\in V(T)$ such that $v\in B_x$.  A path $v_i,\ldots,v_r$ in $H$ is \emph{vertical} if $x_T(v_{i-1})$ is a strict $T$-ancestor of $x_T(v_i)$, for each $i\in\{1,\ldots,r\}$. (Equivalently, $v_0,\ldots,v_r$ is a subsequence of $\pi_1,\ldots,\pi_n$.) Any path in $H$ that is not vertical is \emph{humped}.  This allows us to define the humped us-chromatic number $\hus(H)=\hus(H,\mathcal{T})$ as in \cref{sec:trees}.
%
% \begin{lem}
%     For any $t$-tree $H$, $\uqs(H)\le \binom{t+2}{t}\hus(H)$.
% \end{lem}
%
% \begin{proof}
%     A lemma of \citet{pilipczuk.siebertz:polynomial} (see \cite[Lemma~13]{pilipczuk.siebertz:polynomial-arxiv}) shows that from any vertex $v\in V(H)$, the number of vertical paths of length at most $p$ that end at $v$ is at most $\binom{t+p}{t}$.  Applying this with $p=2$ implies that we can greedily colour $v_1,\ldots,v_n$ to obtain a colouring $\psi:V(H)\to \{1,\ldots,\binom{t+2}{t}\}$ so that the endpoints of any non-trivial vertical path\footnote{A path is non-trivial if its length is at least 1.} of length at most $2$ are assigned different colours.  Let $\varphi:V(H)\to\{1,\ldots,k\}$ be a humped us-colouring of $H$.  As in \cref{product}, the colouring $\varphi':V(H)\to\{1,\ldots,k\}\times \{1,2,3\}$ defined by $\varphi'(v)=(\varphi(v),\varphi(v))$ proves the result.
% \end{proof}
%
% \begin{lem}
%     For any graph $G$, $\uqs(G)\le 3\pw(G)+1$.
% \end{lem}
%
% \begin{proof}[Proof sketch]
%     By induction on $\pw(G)$ starts by showing that, in the base case $\pw(G)=1$ (a caterpillar) the feet can all be coloured 1 and the the spine can be coloured with a repeating sequence $2,3,4,2,3,4\ldots$.
% \end{proof}
%
%
% \begin{obs}\label{colour-partition}
%     Let $G$ be a graph, $X\subset V(G)$ and $\varphi:V(G)\to\N$ be such that
%     \begin{compactenum}
%         \item $\varphi:X\to\N$ is one-to-one;
%         \item $\varphi$ is a us-colouring of $G-X$; and
%         \item $\{\varphi(v):x\in X\} \cap \{\varphi(v):x\in V(G)\setminus X\} =\emptyset$.
%     \end{compactenum}
%     Then $\varphi$ is a us-colouring of $G$.
% \end{obs}
%
% For a node $x\in V(T)$, $T_x$ denotes the subtree of $T$ induced by all $T$-descendants of $x$, including $x$. Let $k$ be an integer such that $n:=|V(H)|\le \sqrt[4]{k!}$. The \emph{rank} of $x$, denoted $\rho_T(x)$ is the largest integer $c$ such that $|T_x| \le \sqrt[4]{k!/c!}$.  Since $\sigma_T(x) \le n \le \sqrt[4]{k!}$, $\rho_T(x)$ is well defined and $\rho_T(x)\in\{1,\ldots,k\}$ for every $x\in V(T)$.
%
% For each $c\in\{1,\ldots,k\}$ let $L_c=\{x\in V(T):\rho_T(x) = c\}$ and let $X_c=\bigcup_{x\in L_c} B_x$.  For each $c\in\{1,\ldots,k\}$, $T[L_c]$ is a forest. Each root node of $T[L_c]$ is called a \emph{$c$-critical node}.  A node that is \emph{critical} if it is $c$-critical for some $c\in\{1,\ldots,k\}$.
%
% \begin{lem}\label{descendant-count}
%     For any node $x\in V(T)$ with $\rho_T(x)=c$, the number of $(c+1)$-critical $T$-descendants of $x$ is $O(\sqrt{c})$.
% \end{lem}
%
% \begin{proof}
%     Since $\rho_T(x)=c$, $|T_x|\le \sqrt[4]{k!/c!}$.  Each $(c+1)$-critical node $y$ has $|T_y|> \sqrt[4]{k!/(c+2)!}$.  Letting $b$ denote the number of $(c+1)$-critical $T$-descendants of $x$ and chaining these inequalities gives
%     \[
%         b\sqrt[4]{k!/(c+2)!} < |T_x| \le \sqrt[4]{k!/c!} \enspace ,
%     \]
%     so $b \le \sqrt[4]{(c+1)(c+2)}=O(\sqrt{c})$.
% \end{proof}
%
% The following lemma illustrates our plan for us-colouring $H$.  Namely, it allows us to focus our remaining effort on finding a us-colouring of the subgraph of $H$ induced by critical nodes that has some additional properties.
%
% \begin{lem}
%     Let $C$ denote the set of critical nodes in $T$ and let $X=\bigcup_{x\in C} B_x$. and let $\varphi: X\to\N$ be a humped us-colouring of $H[X]$ such that, for each $c\in\{1,\ldots,k\}$ and each tree $T'$ in the forest $T[L_c]$, the restriction $\varphi: X\cap \bigcup_{x\in V(T')}B_x$ is one-to-one.  Then $\varphi$ can be completed to a humped us-colouring of $H$ such that $\varphi(v)\in\{-at\log k,\ldots,-1\}$ for every $v\in V(H)\setminus X$.
% \end{lem}
%
% \begin{proof}
%     Consider the graph $H-X$.  The tree decomposition $(B_x:x\in T)$ yields a width-$t$ tree decomposition of $H-X$ in which each component $H'$ is mapped to a subtree $T'$ in the forest $T-C$.  By \cref{descendant-count}, (TODO: State this explicitly) each tree $T'$ in the forest $T-C$ has at $O(\sqrt[4]{k})$ leaves and therefore $\pw(T')=O(\log k)$.  It follows that $\pw(H')=O(t\pw(T'))=O(t\log k)$.  Thus $H'$ can be us-coloured using the colour set $\{-at\log k,\ldots,-1\}$.  By \cref{colour-partition}, this extends $\varphi$ to a us-colouring of $H[V(H')\cup N_H(V(H'))]$.  This graph covers every path $uvw$ of $H$ except those with $u$ and $w$ in different components of $H-X$.  But any such path must have $v\in X$, so $\varphi(v)\ge 0$ and $\varphi(u),\varphi(w)<0$.
% \end{proof}
%
%
%
%
%
%
%
%
%
%
%
%
%
%
%
%
% Key definition:  The \emph{$i$-heavy subtree} of $T$ consists of those nodes $x$ such that the subtree $T_x$ rooted at $x$ has at least $f(i-1)$ nodes.  A heavy subtree will have at most $\alpha(k-i)$ leaves.
%
%
% For each triple of integers $1\le i,t \le k$, $1\le i\le k$, let $f_{k,t}(i)$.
%
%
%
% \section{Applying the Product Structure Theorem}
% \seclabel{products}
%
%
% \section{Conclusions}
% \seclabel{conclusion}
%
% New Stuff:
%
% \begin{lem}\label{path-colour}
%     Let $K$ be a $t$-path, let $\mathcal{P}:=(B_x:x\in V(P))$ be a width-$(t+1)$ path-decomposition of $K$ using the path $P=0,\ldots,m$, let $v_0$ be any vertex in $B_0$, and let $\varphi:B_{0}\setminus\{v_0\}\to\{\alpha\in N:\alpha > 3t\}$ be precolouring of all but one vertex of $B_{0}$.  The $\varphi$ can be extended to a us-colouring of $K$ in which $\varphi(v)\in\{0,\ldots,3t\}$ for each $v\in V(K)\setminus (B_0\setminus v_0)$.
% \end{lem}
%
% \begin{proof}
%     The proof is by induction on $t$. In the base case $t=0$ and $K$ consists of an independent set (TODO: make sure $0$-tree is defined.) and all its uncoloured vertices can be coloured $0$.
%
%     If $t>0$, it is well-known that $K$ contains a ``greedy path'' $w_0,\ldots,w_q$ such that $w_0\in B_0$, $w_q\in B_m$ and with the following property: (*)~there is no pair $1\le i<j-1\le q-1$ such that $w_i$ and $w_j$ are both adjacent to some $v\in V(K)\setminus \{w_0,\ldots,w_q\}$
%
%     For each $i\in\{0,\ldots,q\}$, set $\varphi(w_i)=3t-i\bmod 3$ (unless $i=0$ and $w_0\neq x_0$, in which $\varphi(w_0)$ is already defined.) Now, the graph $K-\{w_0,\ldots,w_q\}$ is a partial $(t-1)$-path which can be augmented to to a $(t-1)$-path $K'$.  Applying induction on $K'$ gives a us-colouring $\varphi:V(K')\to\{0,\ldots,3(t-1)\}$ of $K'$ and it is easy to check that, because of property~(*) this completes $\varphi$ to a us-colouring of $K$.
% \end{proof}
%
% \begin{lem}
%     Let $H$ be a $t$-tree, let $\mathcal{T}:=(B_x:x\in V(T))$ be a canonical path decomposition of $H$ where $T$ has root $r$,  let $T'$ be a subtree of $T$ with root $r'\neq r$ and having at most $\ell$ leaves, let $\kappa = (3t+1)\lceil\log_2(\ell+1)\rceil$, and let $\varphi:B_{r'}\setminus\{r'\}\to \{\alpha\in\N: \alpha > \tau\}$.  Then $\varphi$ can be completed to a us-colouring of the subgraph $H'=H[\bigcup_{x\in V(T')}B_x]$ so that $\varphi(v)\in\{1,\ldots,\tau\}$ for each $v\in V(H')\setminus (B_{r'}\setminus r')$.
% \end{lem}
%
% \begin{proof}
%     The proof is by induction on $\ell$.  When $\ell=1$, $T'$ is a path and $H'$ is therefore a $t$-path.  The result then follows immediately from \cref{path-colouring}.
%
%     Next we consider the case $\ell >1$.
%     Let $T$ be ordered arbitrarily and let $x_1,\ldots,x_\ell$ be the leaves of $T'$ in the order they are encountered by an in in-order traversal.  Consider the path $P$ from the root of $T'$ to $x_{\lceil \ell/2\rceil}$.   Let $H[P]:=H[\bigcup_{x\in V(P)} B_x]$ and apply \cref{path-colouring} to $H[P]$ using the path decomposition $(B_x:x\in V(P))$ to obtain a us-colouring of $H[P]$ in which (after shifting) any newly-coloured vertex is assigned a colour in $\{\kappa - 3t,\ldots,\kappa\}$. Now, $T'-P$ is a forest that can be partitioned into two sets of trees, each of which has at most $\ell/2$ leaves.  We can now apply induction on each tree individually to obtain a us-colouring in which newly coloured nodes are assigned colours in $\{1,\ldots,\kappa-3t\}$.  It is easy to check that this gives a us-colouring of $H$.
%
%     TODO: Check constants.
% \end{proof}
%
% \begin{lem}
%     Let $H$ be a $t$-tree, let $\mathcal{T}:=(B_x:x\in V(T))$ be a canonical path decomposition of $H$,  let $T'$ be a subtree of $T$, let $\mathcal{P}=(P_x:x\in P)$ be a width-$p$ path decomposition of $T'$, let $S:=\bigcup_{x\in V(T')} B_x$  and let $Y:=\{x\in V(T): B_x\cap S\neq\emptyset\}$.
% \end{lem}
%


\bibliographystyle{plainnat}
\bibliography{us}

\end{document}
